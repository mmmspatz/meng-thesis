\chapter{Testing}
The aim of this project was to create a custom coil that performs better than existing arrays in a
particular application. Our primary means of evaluating relative performance is direct comparison of two related
performance metrics: SNR maps and SENSE geometry factor (g-factor) maps. A brief discussion of each follows. For a
complete treatment, see the original SENSE paper: \cite{Pruessmann1999}.

\section{Signal to Noise Ratio Maps}
The MR signal to noise ratio provided by any coil or coil array varies as a function
of space. A a heatmap of SNR in a given plane is a useful visual tool that can be used to judge wheter a coil is well
suited to imaging in a particular region of interest such as the fetal brain or placenta. A SNR map is easy to generate
for a single coil. One can simply run an imaging sequence with intrincicly high SNR twice; once with an initial RF
excitation and once without. The first sequence should produce an image that is dominated by the MR signal, and the
second sequence produces a noise only image. A SNR map in conventionally defined SNR units is obtained by dividing the
first image by the standard deviation of the noise only image.

Generating SNR maps for multi coil arrays is less straigtforward, and depends on the method used to combine data from
indidiual elements. The Roemer paper \cite{Roemer90} describes an optimal way of combining array coil data in the
spatial domain that results in maximum SNR and normalized noise intensity in every voxel of the resulting image.  First,
a complete image is acquired and reconstructed from every element in the array. Next, a sample of pure noise data is
acquired to generate a channel noise correlation matirx $\Psi$, of which a single element $Psi_{ij}$ is the noise
correlation coefficient between receivers $i$ and $j$. Next, the individual coil images are combined on a voxel by voxel
basis. For a single voxel, arrange the values $S_i$ of that voxel in each of the indivudual coil images in a vector $S$.
Similarly, arrange the sensitivities of each coil to the voxel under consideration in a vector $C$.The matrix equation
for the voxel intensity in the optimal SNR uniform noise image is then \ref{eq:I_OPT}. If the individual coil images
have sufficiently high SNR, then $S$ serves as a good approximation of $C$, and equation simplifies to \ref{eq:I_COV}.
If it is assumed that there is no noise correlation between distinct channels, $\Psi$ becomes the identity matrix, and
the optimal SNR formula simplifies to \ref{eq:I_RSOS}. This is equivilent to summing the squared magnitudes of the
uncombined images, then taking the square root of the result.

\begin{equation} \label{eq:I_OPT}
I_{OPT}=\frac{C^H\Psi^{-1}S}{\sqrt{C^H\Psi^{-1}C}}
\end{equation}

\begin{equation} \label{eq:I_COV}
I_{COV}=\sqrt{S^H\Psi^{-1}S}
\end{equation}

\begin{equation} \label{eq:I_RSOS}
I_{RSOS}=\sqrt{S^{H} S}
\end{equation}

TODO:SNR equations

\section{SENSE Geometry Factor}
