\chapter{Figures of Merit}
The aim of this project was to create a custom coil that performs better than existing arrays in a
particular application. Our primary means of evaluating relative performance is direct comparison of two related
performance metrics: SNR maps and SENSE geometry factor (g-factor) maps. A brief discussion of each follows. For a
complete treatment, see the original SENSE paper: \cite{Pruessmann1999}.

\section{Signal to Noise Ratio Maps}
The MR signal to noise ratio provided by any coil or coil array varies as a function
of space. A heatmap of SNR in a given plane is a useful visual tool that can be used to judge whether a coil is well
suited to imaging in a particular region of interest such as the fetal brain or placenta. A SNR map is easy to generate
for a single coil. One can simply run an imaging sequence with intrinsically high SNR twice; once with an initial RF
excitation and once without. The first sequence should produce an image that is dominated by the MR signal, and the
second sequence produces a noise only image. A SNR map in conventionally defined SNR units is obtained by dividing the
first image by the standard deviation of the noise only image.

Generating SNR maps for multi coil arrays is less straightforward, and depends on the method used to combine data from
individual elements. The Roemer paper \cite{Roemer90} describes an optimal way of combining array coil data in the
spatial domain that results in maximum SNR and normalized noise intensity in every voxel of the resulting image.  First,
a complete image is acquired and reconstructed from every element in the array. Next, a sample of pure noise data is
acquired to generate a channel noise covariance matrix $\Psi$, of which a single element $Psi_{ij}$ is the noise
covariance between receivers $i$ and $j$. Next, the individual coil images are combined on a voxel by voxel basis. For a
single voxel, arrange the values $S_i$ of that voxel in each of the individual coil images in a vector $S$.  Similarly,
arrange the sensitivities of each coil to the voxel under consideration in a vector $C$.The matrix equation for the
voxel intensity in the optimal SNR uniform noise image is then \ref{eq:I_OPT}. If the individual coil images have
sufficiently high SNR, then $S$ serves as a good approximation of $C$, and equation simplifies to \ref{eq:I_COV}.  If it
is assumed that there is no noise correlation between distinct channels and that all channels are identically loaded,
$\Psi$ becomes proportional to the identity matrix, and the optimal SNR formula simplifies to \ref{eq:I_RSOS}. This is
equivalent to summing the squared magnitudes of the uncombined images, then taking the square root of the result.

\begin{equation} \label{eq:I_OPT}
I_{OPT}=\frac{C^H\Psi^{-1}S}{\sqrt{C^H\Psi^{-1}C}}
\end{equation}

\begin{equation} \label{eq:I_COV}
I_{COV}=\sqrt{S^H\Psi^{-1}S}
\end{equation}

\begin{equation} \label{eq:I_RSOS}
I_{RSOS}=\sqrt{S^{H} S}
\end{equation}

TODO:SNR equations

\section{SENSE Geometry Factor}

\subsection{k-space sampling for the 2DFT}
The basic Fourier abstraction used in traditional 2D MRI with Cartesian k-space sampling illuminates a fundamental
constraint on imaging speed: k-space traversal speed.

\subsubsection{Sampling Constraints}
The spatial extents of an image dictate the spectral spacing of samples that must be acquired to encode it with the DFT.
\cite{nishimura} .In order to 2D Fourier encode an image of width $FOV_x$ and height $FOV_y$, k-space samples must be
spaced $\frac{1}{FOV_X}$ apart in the $k_x$ direction and $\frac{1}{FOV_Y}$ apart in the $k_y$, direction, as defined in
eq. \ref{eq:fov}.

\begin{equation}\label{eq:fov}
    \begin{aligned}
        FOV_x z &= \frac{1}{\delta k_x}\\
        FOV_y &= \frac{1}{\delta k_y}
    \end{aligned}
\end{equation}

The spatial resolution of an image dictates the extents of spectral data that must be acquired to encode it with the
DFT. In order to 2D Fourier encode an image with voxels $\delta_x$ wide and $\delta_y$ tall, data must be sampled over a
period of approximately $\frac{1}{\delta_x}$ in the $k_x$ direction and $\frac{1}{\delta_x}$ in the $k_x$ direction, as
defined in eq. \ref{eq:spatial_resolution}.


\begin{equation}\label{eq:spatial_resolution}
    \begin{aligned}
        \delta_x &\approx \frac{1}{2k_{xmax}}\\
        \delta_y &\approx \frac{1}{2k_{ymax}}
    \end{aligned}
\end{equation}

It is clear that the number k-space samples needed to resolve an image grows linearly with both FOV and voxel density,
and that the maximum extents of those samples in k-space increases linearly with voxel density. K-space is sampled by
traversing a continuous path that sequentially visits the coordinates of each sample. The maximum speed of k-space
traversal is fundamentally limited by patient safety considerations and technological limitations. So, for a given
maximum speed and optimum path, there is a minimum acquisition time for a fully sampled 2DFT sequence with a given FOV
and spatial resolution.

\subsubsection{Cartesian Under sampling}
An imaging sequence can be accelerated by skipping some points in k-space, then attempting by some means to reconstruct
a complete image from incomplete k-space data. In the case of cartesian sampling for the 2DFT, a natural strategy is to
regularly skip entire lines of k-space data in one dimension.\cite{Pruessmann1999}, for example acquiring every other
line (R=2) or every third line (R=3). The effect of this regular undersampling strategy is to effectively increase
$\delta k$ by a factor of R, and to accordingly decrease $FOV$ by the same factor of R. This causes parts of the object
being imaged that lie outside of the reduced FOV to wrap around the edges of that FOV and alias onto other voxel
positions.

\subsubsection{SENSE}
The Pruessmann paper \cite{Pruessmann1999} introduced the strategy of "Sensitivity Encoding" as a means of undoing the
aliasing induced by undersampling. In an array coil, different receive elements have different spatial sensitivity
profiles. These varying sensitivities provide different "views" of the same aliased image, and this extra information is
used to unfold the aliased image on a pixel by pixel basis.

Consider an array with $n_C$ individual elements, each with a unique view of a voxel location that, in the reduced FOV
image, represents a superposition of $n_P$ physical voxel locations. A sensitivity matrix $S$ is constructed to describe
the sensitivity of each receive element to each of the voxel positions that are aliased into a single pixel in the final
image. An individual element $S_{\gamma,\rho}$ in this matrix represents the sensitivity of receive element number
$\gamma$ to superimposed pixel $\rho$. The values of the voxel in the reduced FOV images from each receive element are
arranged in a vector of length $n_C$: $\vec{a}$. Similar to in the previous section, a $n_C \times n_C$ receiver noise covariance
matrix. The unfolding matrix is then $U$ (eq. \ref{eq:unfolding}), and the vector of image values in the $n_p$
voxels of the unaliased image the are estimated by $\vec{v} = U\vec{a}$. This unfolding process is repeated for every
voxel in the reduced FOV image.

\begin{equation}\label{eq:unfolding}
    U = (S^H\Psi^{-1}S)^{-1}S^H\Psi^{-1}
\end{equation}

\subsubsection{SENSE g-factor}
By accelerating an image acquisition by a factor of $R$, we expect that the SNR of the resulting image must
correspondingly decrease by a factor of $\frac{1}{\sqrt{R}}$ \cite{nishimura}. Usually, though, imperfect conditioning
of the sensitivity matrix results in a a steeper SNR penalty. The extent to which SENSE underperforms the best case SNR
penalty is characterized by the geometry factor, or g-factor, defined in eq. \ref{eq:SENSE_g}. The g-factor varies as a
function of space, and is defined for each voxel in the full FOV image. It is called the geometry factor because it
depends largely on the geometry of the receive array. With more independent views of an aliased pixel are available, a
higher acceleration factor can be attempted without poor conditioning amplifying noise to an unacceptable degree.

\begin{equation}\label{eq:SENSE_g}
    g_{SENSE} = \frac{1}{\sqrt{R}} \cdot \frac{SNR_{FULL}}{SNR_{ACCEL}}
\end{equation}

Plots of 2D SENSE g-factor maps in a given slice are a useful graphical tool for assessing the ability of an array to support
a given acceleration factor along a particular dimension (or, along two dimensions simultaneously.) SENSE g-factor is
widely used to compare the acceleration capability of different coil arrays because it is straightforward to compute and
gives insights into the fundamental ability of an array to enable acceleration using any method. Still, it is important
to note that unique formulations of g-factor have developed for other acceleration methods, such as SMASH and GRAPPA \cite{Breuer2009}.
