\newcommand{\uvec}[1]{\bm{\hat{#1}}}
\newcommand{\bvec}[1]{\bm{\vec{#1}}}

\chapter{Background}
Magnetic resonance imaging is, as the name suggests, enabled by the phenomenon of magnetic resonance.

\section{Origin of MR signal}

\subsection{Nuclear Spins}
Spin is a quantum mechanical property of a particle that describes its intrinsic angular momentum. Atoms with an odd
number of protons or neutrons possess net nuclear spin, and therefore have mutually aligned angular momentum $\bvec{S}$
and magnetic dipole moment $\bvec{\mu}= \gamma \bvec{S}$, where $\gamma$ is the gyromagnetic ratio: a known constant
defined for every nucleus. Hydrogen (\ce{^{1}_{1}H}) is such an atom, having one proton and no neutrons. Along a given
axis, hydrogen spins are quantized to $\pm \frac{\hbar}2$.  Therefore, hydrogen dipole moments are likewise quantized to
eq. \ref{eq:dipole}.

\begin{equation}\label{eq:dipole}
    \mu = \pm \gamma \frac{\hbar}{2}
\end{equation}

\subsection{Spins in the Presence of External Magnetic Field}
In the presence of a strong polarizing field pointing in the $\uvec{k}$ direction and with magnitude $B_0$, the
potential energy of a spin with dipole moment $\bvec{\mu}$ is the dot product between the two vectors (eq.
\ref{eq:spin_energy}). Because $\mu$ is quantized, there is a gap $\Delta E$ between realizable energy states (eq.
\ref{eq:delta_E}).

\begin{equation}\label{eq:spin_energy}
    E = \bvec{\mu} \cdot \bvec{B_0}
\end{equation}

\begin{equation}\label{eq:delta_E}
    \Delta E = \gamma \hbar B_0
\end{equation}

\subsubsection{Polarization}
Spins tend to settle in the low energy state, pointing in the direction of $\bvec{B_0}$. However, at room temperature,
thermal energy vastly exceeds the energy gap between the two states, and the ratio of spins aligned with $\bvec{B_0}$
($n_+$) to those anti aligned ($n_-$) is described by the Boltzmann distribution in eq. \ref{eq:boltzmann}. For hydrogen
($\gamma = \frac{42.58}{2\pi} \frac{MHz}{T}$) at room temperature ($T=273 K$) at a field strength of $3T$, $\Delta E =
5.283 \times 10^{-7} eV$, and $\frac{n_-}{n_+} \approx (1 - 2.25 \times 10^{-5})$. This relatively tiny fraction of
excess spin polarization is the source of the NMR signal. Luckily, there are $3.3428 \times 10^{23}$ protons in a gram
of water, resulting in $7.5 \times 10^{18}$ aligned spins per gram under the previously stated conditions. Since living
things tend to contain mostly water, the proposal of MRI is still promising.

\begin{equation}\label{eq:boltzmann}
    \frac{n_-}{n_+} = \exp(-\frac{\Delta E}{k T}) = \exp(-\frac{\gamma \hbar B_0}{k T})
\end{equation}

In a large population of spins, the net magnetic dipole moment per unit volume is termed $\bvec{M}$. In equilibrium, it
is the product of the volumetric spin density $N$, the magnitude of a single spin dipole moment $\mu$, and the excess
fraction of aligned spins. It points in the same direction as $\bvec{B_0}$ and has magnitude $M_0$. Using the first two
terms of a taylor series expansion of the Boltzmann distribution, we can approximate $M_0$ as eq. \ref{eq:M}.

\begin{equation}\label{eq:M}
    M_0 = N \cdot \mu \cdot (1-\exp(-\frac{\Delta E}{k T})) \approx \frac{N \gamma^2 \hbar^2 B_0}{2 k T}
\end{equation}

\subsection{Spin Dynamics}
In equilibrium, $\bvec{M}$ comes to point in the direction $\bvec{B_0}$ with magnitude $M_0$. But the next step will be
to tip $\bvec{M}$ off of the $z$ axis so that it has a component in the $x$-$y$ plane. The observed behavior will then
be time varying.

\subsubsection{Precession}
A single magnetic dipole $\bvec{\mu}$ with mutually aligned angular momentum $\bvec{S}$ placed in an external magnetic
field $\bvec{B}$ will experience a torque $\bvec{\mu} \times \bvec{B}$. Multiplying this torque by $\gamma$ gives an
expression for the time rate of change of $\bvec{\mu}$, eq. \ref{eq:dipole_precession}. Eq. \ref{eq:dipole_precession}
shows $\bvec{\mu}$ moving in a direction perpendicular to both itself and $\bvec{B}$, i.e. precessing about $\bvec{B}$.
The frequency of this precession is $\omega_L$ in eq. \ref{eq:precession_freq}.

\begin{equation}\label{eq:dipole_precession}
    \frac{d \bvec{\mu}}{dt} = \bvec{\mu} \times \gamma \bvec{B}
\end{equation}

\begin{equation}\label{eq:precession_freq}
    \omega_L = \gamma \cdot B
\end{equation}

A population of dipole moments precessing in synchrony gives a net magnetization $\bvec{M}$ that also precesses at
$\omega_L$.

\subsubsection{Longitudinal Relaxation}
Define the component of $\bvec{M}$ that is parallel to $\bvec{B_0}$ as $\bvec{M_z}$. In equilibrium $|\bvec{M_z}| =
M_0$, but immediately following the tipping of $\bvec{M}$ into the $x$-$y$ plane by an angle $\alpha$ at time $t=0$,
$\bvec{M_z}$ is reduced to a value shown in eq. \ref{eq:mz_postflip}.

\begin{equation}\label{eq:mz_postflip}
    \bvec{M_z}(t=0+) = \cos(\alpha) \cdot \bvec{M_z}(0-)
\end{equation}

$\bvec{M_z}$ then begins to exponentially recover to its equilibrium magnitude of $M_0$. The time constant associated
with this longitudinal recovery is termed $T_1$. $T_1$ is dependent on the tissue or material being imaged, but also has
a positive dependence on $B_0$ field strength.

\begin{equation}\label{eq:t1_decay}
    \bvec{M_{z}}(t) = \uvec{k} M_0 + (\bvec{M_{z}}(0) - \uvec{k} M_0) \cdot \exp(-\frac{t}{T_1})
\end{equation}

\subsubsection{Transverse Relaxation}
Define the component of $\bvec{M}$ that is perpendicular to $\bvec{B_0}$ as $\bvec{M_{xy}}$. In equilibrium,
$\bvec{M_{xy}} = 0$.  Immediately following the tipping of $\bvec{M}$ into the $x$-$y$ plane by an angle $\alpha$ at
time $t=0$, $\bvec{M_{xy}}$ is as shown in eq. \ref{eq:mxy_postflip}. Individual spins begin to dephase as soon as they
are tipped into the $x$-$y$ plane, and so the net transverse magnetization $\bvec{M_{xy}}$ experiences exponential
decay, as shown in eq. \ref{eq:t2_decay}. The time constant associated with this transverse decay is termed $T_2$, and
is a property of the tissue or material being imaged.

\begin{equation}\label{eq:mxy_postflip}
    \bvec{M_{xy}}(t=0+) = \sin(\alpha) \cdot \bvec{M_z}(0-)
\end{equation}

\begin{equation}\label{eq:t2_decay}
    \bvec{M_{xy}}(t) = \bvec{M_{xy}}(0) \cdot \exp(-\frac{t}{T_2})
\end{equation}

\subsection{Bloch Equation}
Assembled together, the spin dynamics described above form the Bloch equation, shown in eq. \ref{eq:bloch}. The complete
Bloch equation describes the behavior of spins in a generalized external magnetic field $\bvec{B}$ that is the sum of
the main field $B_0$, the RF field $B_1$, and the linearly varying gradient fields $G$. 

\begin{equation}\label{eq:bloch}
    \frac{d \bvec{M}}{dt} = \bvec{M} \times \gamma \bvec{B} - \frac{M_x \uvec{i} + M_y \uvec{j}}{T_2} - \frac{(M_z - M_0)
    \uvec{k}}{T_1}
\end{equation}

\subsection{RF Excitation}
In the previous section, we covered what happens when $\bvec{B} = B_0 \uvec{k}$; the steady state solution to the Bloch
equation is $\bvec{M} = M_0 \uvec{k}$, where $M_0$ is as defined in eq. \ref{eq:M}.  Now consider the case where
$\bvec{B}$ is as in eq. \ref{eq:B_rot}. A transverse component rotating at a frequncy $\omega_R$ and with time varying
amplitude $B_1(t)$ has been added. 

\begin{equation}\label{eq:B_rot}
    \bvec{B} = B_0 \uvec{k} + B_1(t) \cdot (\cos(\omega_R t) \uvec{i} - \sin(\omega_R t)\uvec{j})
\end{equation}

\section{Signal Equation}

The NMR signal arises from the precessing transverse component of the magnetization vector. As we begin to focus solely
on $\bvec{M_{xy}}$, I will follow the convention used by Nishimura \cite{nishimura} and define a new variable $M$ that
represents the transverse magnetization as a function of location and time with single complex number.

\begin{equation}\label{eq:M_complex}
    M(\bvec{r},t) \equiv |\bvec{M_x}| + j |\bvec{M_y}|
\end{equation}

\subsection{Gradient Fields}

Spatial encoding in traditional MRI is achieved by applying linearly varying gradient fields on top of the main field,
as shown in eq. \ref{eq:gradients}. Just like $\bvec{B_0}$, $\bvec{G}$ points in the $\uvec{k}$ direction. Unlike
$\bvec{B_0}$, $\bvec{G}$ varies as a function of space and time.

\begin{equation}\label{eq:gradients}
    \bvec{B} = (B_0 + G(\bvec{r},t))\uvec{k}
\end{equation}

After an initial flip resulting in a spatial magnetization distribution $M(\bvec{r})$, spins begin precess under the
influence of $G$, as described in \ref{eq:phase_enc}. After a time $t$, spins at a location $\bvec{r}$ have accrued
excess phase in proportion to the time integral of $G(\bvec{r},t)$.

\begin{equation}\label{eq:phase_enc}
    M(\bvec{r},t) = M(\bvec{r,t=0}) e^{-j \gamma B_0 t} exp(-j \gamma \int_0^t G(\bvec{r},\tau) d \tau)
\end{equation}






\section{Imaging Methods}
\subsection{RF Excitation}
\subsection{Linear Gradient Fields}
\subsection{Imaging with the 2DFT}

\section{Noise in MRI}
