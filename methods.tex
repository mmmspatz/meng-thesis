\chapter{Methods}

\section{Bench Tests}
\subsection{Intercoil Coupling and Reflection Coefficient Measurements}
A network analyzer connected directly to loop output ports was used to measure coupling between neighboring coil
elements ($S_{12}$), and to measure individual coil output reflection coefficients ($S{11}$). For each of theses tests,
the output power of the network analyzer was reduced to -25dBm and the coil was appropriately loaded by a test phantom.
This test configuration was used during the iterative geometric decoupling adjustment process, where the loops are
manually bent and reconfigured to minimize neighbor-to-neighbor coupling.

\subsection{Loop Detuning and Preamp Decoupling Verification}
The active detune capability and preamp decoupling performance were  tuned and characterized using a pair of decoupled
($S_{12}<70dB$) inductive probes loosely coupled to the loop under test. In this arrangement, the $S_{12}$ measurement
between the two probes is directly proportional to the current flowing in the loop \cite{Reykowski1995}. Both the active
detune capability and preamplifier decoupling strategy work by introducing a second resonance near the loop resonance
frequency, which "splits" the loop resonance into two peaks above and below the initial resonance frequency. The null
between these peaks is moved to $\omega_L$ by adjustment of a tunable component. For the active detune strategy, this
tunable component is an adjustable air-core inductor ($L_{TRAP}$ in fig. \ref{fig:loop_schematic}). For preamp
decoupling, it is a trim capacitor ($C_M$ in fig.  \ref{fig:loop_schematic}).

\section{MRI Data Acquisition and Reconstruction}
Images of the pregnant abdominal phantom were acquired for using both the 64 channel fetal coil and the standard
combination of 16 channels from a 32 channel spine array and all channels from an 18 channel flexible body array. Three
orthogonal slices (transverse, coronal, and sagital) intersecting the fetal brain compartment were carefully duplicated
using both array configurations. An extremely high SNR PD weighted 2DGRE sequence ($TR=3500ms, TE=4ms, FA=45^{\circ},
FOV=400mm, \delta \approx 2mm \times 2mm \times 7mm, BW=180Hz/px$) was chosen so that each of the uncombined coil images
has decent SNR ($>20$) in the deep central region of the fetal brain compartment, and thus can be taken as an accurate
approximation of a coil sensitivity map \cite{Roemer90}. The same sequence was run with the reference TX voltage set to
$0V$ to acquire noise-only data for the generation of a noise covariance matrix $\Psi$. Data were acquired on a 3T
Magnetom Skyra System (Siemens Healthcare, Erlangen, Germany).

Covariance weighted SNR maps and SENSE g-factor maps were computed offline using the resulting raw data. For the fetal
coil, single channel SNR maps were generated by dividing each of the 64 uncombined coil images by the standard deviation
of a corresponding noise only image. The mean value of each single channel SNR map was computed inside an ROI ($A=29$
voxels) in the fetal brain region of the anthropomorphic phantom as a means of assessing the relative importance of each
component in the array geometry.


