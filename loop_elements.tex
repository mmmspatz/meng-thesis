\chapter{Loop Elements}

\section{Loop Matching}
\subsection{Loop Model}

The series resnant loop antenna (Figure \ref{fig:loop_model} A) is constructed to have the highest possibles unloaded Q
($Q_{UL}$). When the loop is placed in close proximity to a conductive object (such as a body part, a circuit board, or
a neighboring loop), current flowing in the loop induces eddy currents in the conductive object. If the conductivity of
the object is finite, then these eddy currents will deposit energy in the object, and the Q of the loop will be reduced.
This effect can be modelled by a series resistance RL in the loop circuit (LOOP FIGURE B). Assuming that external causes
of loop loading dominate over wire and component resistances, the exact value of RL can be determined by measuring the Q
of the loaded loop ($R_=L\frac{\sqrt{\frac{L}{C}}}{Q}$).

The Vendor supplied preamplifier is designed to achieve minimum noise figure when presented with a purely real 50 ohm
load at its input. Therefore, a matching network is needed to transform 

1. Resonant loop, CS CP

\begin{figure}
\vspace{2.4in}
\import{figures/}{loop_model.pdf_tex}
    \caption{A) Bare loop circuit model B) Loop model with output port C) impedance model}
\label{fig:loop_model}
\end{figure}

\section{Description of micro-optimization}\label{ch1:opts}
\subsection{Post Multiply Normalization}
% This is an example of how you would use tgrind to include an example
% of source code; it is commented out in this template since the code
% example file does not exist.  To use it, you need to remove the '%' on the
% beginning of the line, and insert your own information in the call.
%
%\tagrind[htbp]{code/pmn.s.tex}{Post Multiply Normalization}{opt:pmn}
\subsection{Block Exponent}

