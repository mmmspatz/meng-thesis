\chapter{Loop Elements}

The same basic loop circuit, shown in figure \ref{fig:loop_schematic}, is duplicated 64 times to create the full array.
The components appearing inside the dashed green box exist on a small FR4 circuit board termed the "feed point board."
Here, I will discuss the purpose and function of each part of this circuit. All capacitors used in the loop circuit are
Voltronics Series 11 non-magnetic.

\section{Active Detuning}
The loop is a resonant circuit that is strongly coupled to the volume surrounding it. It is necessary to spoil this
resonance during the high power RF transmit pulses so that excessive currents are not induced in the loop. Such
unintended energy deposition could adversely affect transmit homogeneity, damage the array, or create a safety hazard.
Selective detuning is achieved by switching an inductor across one of the loop capacitors, creating a parallel
resonant tank that behaves as an open circuit in the loop at $\omega_L$.

A DC bias current of about $120mA$ is injected on the line marked BIAS in figure \ref{fig:loop_schematic}. This bias
current flows through a PIN diode $D_1$ (MACOM MA47461F-1072), creating an RF short and effectively switching $L_{TRAP}$ across $C_{S3}$.
$L_{TRAP}$ is an adjustable air core inductor, and is hand tuned to resonate with $C_{S3}$ at precisely $\omega_L$. The
parallel resonant circuit thus formed creates a virtual open circuit in the loop, preventing current from circulating in
the loop. As soon as the bias current is removed and the diode recovers, the trap is disabled and the loop once again
becomes tuned. The $Q$ of this active detuning trap is kept high by using a relatively high value high Q inductor for
$L_{TRAP}$. In this case, we used a $~60nH$ hand wound inductor made with 16 AWG enameled copper wire.

The bias current is injected through a bias tee formed by $L_{BIAS}$ ($3.3uH$ Coilcraft) and a $1nF$ capacitor to prevent leakage or
injection of RF signal from/to the loop on the bias line.

\section{Passive Detuning}
The active detuning strategy is sufficient for assurance of image quality and protection from hardware damage, but a
passive method is required to ensure patient safety in the event of an electrical failure. The crossed diode pair $D_2$
(BAV99) clamps the voltage across $C_{S3}$ and $L_{TRAP}$ to safe levels, passively enabling the trap if the energy stored in
the loop gets too high. Other designs might also include RF fuses in the loop circuit, but the loops employed in this
array are all small enough that this was not necessary.

\begin{figure}
    \centering
    \input{figures/loop_diagram.pdf_tex}
    \caption{Complete loop circuit schematic}
    \label{fig:loop_schematic}
\end{figure}

\section{Loop Model}
The essence of the wire loop receive elements used in this array is a damped series resonant circuit, shown in figure
\ref{fig:loop_model}A. The loop has a distributed inductance by nature of its geometry, and is broken at regular
intervals by discrete capacitors.  Wire and component resistances and (more importantly) inductive coupling to adjacent
conductive materials reduces the loop $Q$ to a finite value. This effect is modeled by a series resistor with value
$R_{LOAD}=\sqrt{\frac{L}{C}}\cdot\frac{1}{Q}$

\section{Complete Loop Circuit}
Figure \ref{fig:loop_model}B shows the creation of an output port in the loop circuit. The total loop capacitance is
split into $C_P$, across which the output port is formed, and $C_S$. A series capacitor $C_M$ is added to one terminal
of the output port. In the first part of the following analysis, grouping components values into the block impedances
defined in eq. \ref{eq:block_impedances} and fig. \ref{fig:loop_model}C results in clearer and more compact
expressions, so I have made that substitution.

\begin{figure}
    \centering
    \input{figures/loop_model.pdf_tex}
    \caption{Loop circuit models.}
    \label{fig:loop_model}
\end{figure}

\begin{equation}\label{eq:block_impedances}
    \begin{aligned}
        X_S(\omega) &= \omega L_{LOOP} - \frac{1}{\omega C_S}\\
        X_P(\omega) &= \frac{-1}{\omega C_P}\\
        X_M(\omega) &= \frac{-1}{\omega C_M}\\
    \end{aligned}
\end{equation}

\section{Loop Circuit Analysis}
The loop circuit has an input impedance of $Z_{IN}$ at its port, as defined in equation \ref{eq:Z_IN}. This impedance can
be split into its real and imaginary parts, $R_{IN}$ and $X_{IN}$, shown in equations \ref{eq:R_IN} and \ref{eq:X_IN}.

\begin{equation} \label{eq:Z_IN}
    \begin{aligned}
        Z_{IN}&=(jX_p(\omega))\parallel(jX_S(\omega)+R_{LOAD})+jX_M(\omega)\\
        &= \frac{j X_P(\omega) (j X_S(\omega) + R_{LOAD})}{j (X_P(\omega) + X_S(\omega)) + R_{LOAD}} + j X_M(\omega)
    \end{aligned}
\end{equation}

\begin{equation} \label{eq:R_IN}
    R_{IN}=Re(Z_{IN})=\frac{{X_P(\omega)}^2 R_{LOAD}}{{R_{LOAD}}^2+(X_P(\omega)+X_S(\omega))^2}
\end{equation}

\begin{equation} \label{eq:X_IN}
    X_{IN}= Im(Z_{IN}) = \frac{X_P(\omega) ({R_{LOAD}}^2 +
    X_S(\omega)(X_P(\omega)+X_S(\omega)))}{{R_{LOAD}}^2+(X_P(\omega)+X_S(\omega))^2}+X_M(\omega)
\end{equation}

\section{Loop Component selection}

\subsection{Loop Circuit Considerations}
\subsubsection{Minimizing Preamp Noise Figure}
The vendor supplied preamplifier is designed to achieve minimum noise figure when presented with a purely real
$50\Omega$ load at its input. Therefore, component values should be selected such that $R_{IN}=50\Omega$ and
$X_{IN}=0\Omega$. Call this optimal input impedance $Z_{IN_{OPT}}$.

\subsubsection{Preamp Decoupling}
Preamp decoupling is usually achieved by resonating a capacitor in the loop ($C_P$ in figure \ref{fig:loop_model}) with
an inductor in series with one terminal of the output port (in the same position as $C_M$ in figure
\ref{fig:loop_model}) through the input of the preamplifier. In our case, however, the inductance is integrated into the
preamplifier itself. I measure the inductance of the preamplifer input to be roughly $130nH$ at $123.25 MHz$. The
details of the preamp topology are unavailable to me, so I simply consider it to have an impedance of $Z_{PRE}=j\omega
L_{PRE}$, which is transformed to ${Z_{PRE}}'$ (as shown in equation \ref{eq:Z_PREAMP}) by the short length of coaxial
cable (with characteristic impedance $Z_0$) connecting the preamp to the loop.

\begin{equation} \label{eq:Z_PREAMP}
    {Z_{PRE}}'(\omega)=Z_0 \cdot \frac{Z_{PRE}(\omega)-j Z_0 \cdot \tan(2\pi\cdot\frac{L_{COAX}}{\lambda})}{Z_0 - j
    Z_{PRE}(\omega) \cdot \tan(2\pi\cdot\frac{L_{COAX}}{\lambda})}
\end{equation}

\begin{equation} \label{eq:X_PREAMP}
    {X_{PRE}}'(\omega)=Im({Z_{PRE}}'(\omega))
\end{equation}
    
In any case, preamp decoupling is achieved when $C_P$, $C_M$, and the transformed preamp input impedance resonate
together, as defined in equation \ref{eq:preamp_decoupling_condition}.

\begin{equation} \label{eq:preamp_decoupling_condition}
    X_P(\omega) + X_M(\omega) +{X_{PRE}}'(\omega) = 0
\end{equation}

\subsection{Loop Resonance}
I define loop resonance as occurring when $X_P + X_S = 0$, at a frequency of $\omega_0$ (eq. \ref{eq:w0}). In this case,
the equations for $R_{IN}$ and $X_{IN}$ simplify to equations \ref{eq:R_IN_RES} and \ref{eq:X_IN_RES} respectively. One
can begin to see how components could be selected to set $R_{IN}\big|_{\omega=\omega_0}=Re(Z_{IN_{OPT}})$, but with the
topology we've selected it is impossible to achieve $X_{IN}\big|_{\omega=\omega_0}=0$. You could do both if you changed
$C_M$ to an inductor, but then it becomes impossible to meet the preamp decoupling condition (eq.
\ref{eq:preamp_decoupling_condition}).


\begin{equation} \label{eq:w0}
    \omega_0 = \sqrt{\frac{C_P+C_S}{C_P C_S L_{LOOP}}}
\end{equation}

\begin{equation} \label{eq:R_IN_RES}
    R_{IN}\big|_{\omega=\omega_0}= \frac{C_S L_{LOOP}}{R_{LOAD} C_P (C_P + C_S)}
\end{equation}

\begin{equation} \label{eq:X_IN_RES}
    X_{IN}\big|_{\omega=\omega_0}=-\sqrt{\frac{C_S L_{LOOP}}{C_P (C_P + C_S)}} \cdot \frac{C_M + C_P}{C_M}
\end{equation}

\subsection{Off resonance behavior}
It is not necessary that the loop be tuned to resonate precisely at the frequency of interest. The position of
$\omega_0$ relative to that of the Lamor frequency $\omega_L$ is another variable that we can manipulate to achieve the
desired loop circuit characteristics.

\subsubsection{Capacitive Divider Ratio}
Consider choosing $C_P$ and $C_S$ to achieve loop resonance at a frequency a factor $\alpha$ away from $\omega_L$, so
that $\omega_0 = \alpha\cdot\omega_L$. We can quickly eliminate $C_S$ from eq. \ref{eq:w0}. Whatever $C_P$, $\alpha$, and
$\omega_L$ we choose will imply eq. \ref{eq:CS_alpha} for $C_S$.

\begin{equation} \label{eq:CS_alpha}
    C_S(C_P,\alpha,\omega_L) = \frac{C_P}{C_P L_{LOOP} \alpha^2 {\omega_L}^2 -1}
\end{equation}

Plugging eq. \ref{eq:CS_alpha} into eq. \ref{eq:R_IN}, then solving $R_{IN} = Z_{IN_{OPT}}$ for $C_P$, we get eq.
\ref{eq:CP_alpha}.

\begin{equation} \label{eq:CP_alpha}
    C_P(\alpha,\omega_L) = \frac{1}{\omega_L}\sqrt{\frac{R_{LOAD}}{Z_{IN_{OPT}} ( {R_{LOAD}}^2 + (\alpha^2-1)^2 {\omega_L}^2
    {L_{LOOP}}^2)}}
\end{equation}

\subsubsection{Choosing $\alpha$ and $C_M$}
The only remaining free parameters are $\alpha$ and $C_M$.  They must be chosen to achieve purely real input impedance
($X_{IN} = 0$) and preamp decoupling ($X_P(\omega_L) + X_M(\omega_L) +{X_{PRE}}'(\omega_L) = 0$). Plugging in $C_S$ and
$C_P$ from above into these two requirements results in a system of equations \ref{eq:syseq}. Constraining $\alpha$ to
be positive, this system has single solution: $\alpha$ as in eq. \ref{eq:alpha} and $C_M$ as in eq.
\ref{eq:component_values}.

\begin{equation}\label{eq:syseq}
\begin{cases}
    \frac{-1}{\omega_L C_M} + \frac{(\alpha^2-1) \omega_L L_{LOOP} Z_{IN_{OPT}}}{R_{LOAD}} - \frac{1}{\omega_L
    C_P(\alpha,\omega_L)} = 0\\

    \frac{-1}{\omega_L C_M} + \omega_L L_{PRE} - \frac{1}{\omega_L C_P(\alpha, \omega_L)} = 0 

\end{cases}
\end{equation}

\begin{equation}\label{eq:alpha}
    \alpha = \sqrt{1 + \frac{L_{PRE} R_{LOAD}}{L_{LOOP} Z_{IN_{OPT}}}}
\end{equation}

\subsection{Optimal Component values}
With $\alpha$ uniquely determined, $C_M$, $C_P$, and $C_S$ are each fully constrained. Our final component value formulas
are shown in eq. \ref{eq:component_values}.

\begin{equation}\label{eq:component_values}
\begin{aligned}
        C_M &= \frac{1}{\omega_L} \cdot \frac{1}{\omega_L L_{PRE} - \sqrt{\frac{R_{LOAD}}{Z_{IN_{OPT}}} ({\omega_L}^2 {L_{PRE}}^2 + Z_{IN_{OPT}}^2)}}\\
        C_P &= \frac{1}{\omega_L} \cdot \frac{1}{\sqrt{\frac{R_{LOAD}}{Z_{IN_{OPT}}} ({\omega_L}^2 {L_{PRE}}^2 + Z_{IN_{OPT}}^2)}}\\
        C_S &= \frac{1}{\omega_L} \cdot \frac{1}{\omega_L(L_{PRE} \frac{R_{LOAD}}{Z_{IN{OPT}}} + L_{LOOP}) - \sqrt{\frac{R_{LOAD}}{Z_{IN_{OPT}}} ({\omega_L}^2 {L_{PRE}}^2 + Z_{IN_{OPT}}^2)}}
\end{aligned}
\end{equation}

\subsection{Verifying Optimal Component Values}
Figure \ref{fig:impedance_plot} illustrates the success of this component selection strategy in simultaneously achieving
optimal input impedance and preamp decoupling. The impedance plots shown are based on estimated parameters of a loop in
the belly panel. Numerical values for this simulation are summarized in tables \ref{tab:sim_independent} and
\ref{tab:sim_dependent}.

\newpage
\begin{figure}
    \centering
    \input{figures/impedance_plot.pdf_tex}
    \caption{Loop impedances vs. frequency with optimal component values.}
    \label{fig:impedance_plot}
\end{figure}

\begin{table}[]
\centering
\caption{Loop simulation parameters}
\label{tab:sim_independent}
\begin{tabular}{|l|l|}
\hline
Parameter  & Value       \\
\hline
$R_{LOAD}$ & $10\Omega$  \\
$L_{LOOP}$ & $247nH$     \\
$L_{PRE}$  & $130nH$     \\
$Z_{IN_{OPT}}$  & $50\Omega$     \\
$\omega_L$ & $123.25MHz$ \\
\hline
\end{tabular}
\end{table}

\begin{table}[]
\centering
\caption{Calculated optimal component values}
\label{tab:sim_dependent}
\begin{tabular}{|l|l|}
\hline
Parameter & Value       \\
\hline
$C_M$     & $25.8pF$  \\
$C_P$     & $25.8pF$     \\
$C_S$     & $8.0pF$     \\
$\alpha$  & $1.051$ \\
\hline
\end{tabular}
\end{table}
