\chapter{Loop Elements}

\section{Loop Model}
The essence of the wire loop receive elements used in this array is a damped series resonant circuit, shown in figure
\ref{fig:loop_model} A. The loop has a distributed inductance by nature of its geometry, and is broken at regular
intervals by discrete capacitors.  Wire and component resistances and (more importantly) inductive coupling to adjacent
conductive materials reduces the loop Q to a finite value. This effect is modelled by a series resistor with value
$R_{LOAD}=\sqrt{\frac{L}{C}}\cdot\frac{1}{Q}$

\section{Complete Loop Circuit}
Figure \ref{fig:loop_model} B shows the creation of an output port in the loop circuit. The total loop capacitance is split into $C_P$, 
across which the output port is formed, and $C_S$. A series capacitor $C_M$ is added to one terminal of the output port.
For the purpose of further analisys, it is convenient to lump component impedances together into three blocks: $Z_P$ (parallel
imedance), $Z_S$ (series impedance), and $Z_M$ (matching impedance). These impedances are defined in figure \ref{fig:loop_model} C.

\section{Loop Circuit Analisys}
The loop ciruit has an input inpedance of $Z_{IN}$ at its port, as defined in equation \ref{eq:Z_IN}. This impedance can
be split into its real and imaginary parts, $R_{IN}$ and $X_{IN}$, shown in equations \ref{eq:R_IN} and \ref{eq:X_IN}.

\begin{equation} \label{eq:Z_IN}
    Z_{IN}=(jX_p)\parallel(jX_S+R_{LOAD})+jX_M = \frac{j X_P (j X_S + R_L)}{j (X_P + X_S) + R_L} + j X_M
\end{equation}

\begin{equation} \label{eq:R_IN}
    R_{IN}=Re(Z_{IN})=\frac{{X_P}^2 R_L}{{R_L}^2+(X_P+X_S)^2}
\end{equation}

\begin{equation} \label{eq:X_IN}
    X_{IN}= Im(Z_{IN}) = \frac{X_P ({R_L}^2 + X_S(X_P+X_S))}{{R_L}^2+(X_P+X_S)^2}+X_M
\end{equation}

\section{Loop Component selection}
\subsection{Loop Circuit Considerations}
\subsubsection{Minimizing Preamp Noise Figure}
The vendor supplied preamplifier is designed to achieve minimum noise figure when presented with a purely real 50 ohm 
load at its input. Therefore, component values should be selected such that $R_{IN}=50\Omega$ and $X_{IN}=0\Omega$.
\subsubsection{Preamp Decoupling}
Preamp decoupling is typicall achieved by resonating a capacitor in the loop ($C_P$ in figure \ref{fig:loop_model}) with
an inductor in series with one terminal of the output port (in the same position as $C_M$ in figure
\ref{fig:loop_model}) through the input of the preamplifier. In our case, the inductance is integrated into the
preamplifier itself. I measure the inductance of the preamplifer input to be roughly $130nH$ at $123.25 MHz$. The
details of the preamp topology are unavailible to me. I simply consider it to have an impedance of $Z_{PREAMP}$, which
is transformed to ${Z_{PREAMP}}'$ (as shown in equation \ref{eq:Z_PREAMP}) by the short length of coaxial cable
connecting the preamp to the loop.


\begin{equation} \label{eq:Z_PREAMP}
    {Z_{PREAMP}}'=Z_0 \cdot \frac{Z_{PREAMP}-j Z_0 \cdot \tan(2\pi\cdot\frac{L_{COAX}}{\lambda})}{Z_0 - j Z_{PREAMP} \cdot
    \tan(2\pi\cdot\frac{L_{COAX}}{\lambda})}
\end{equation}
\begin{equation} \label{eq:X_PREAMP}
    {X_{PREAMP}}'=Im({Z_{PREAMP}}')
\end{equation}
    
In any case, preamp decoupling is achieved when $X_P+X_M+{X_{PREAMP}}'=0$
