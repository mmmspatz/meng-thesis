\chapter{Discussion}
The modest SNR improvement seen in unaccelerated imaging is likely attributed to the deep (nearly central) location of
the fetal head where coil arrays with more than 16 elements already approach the ultimate SNR available
\cite{Wiesinger2004}. Nonetheless, because the SNR is achieved with array elements with higher spatial frequency
content sensitivity profiles, there is an improvement in acceleration capability.

The anterior and posterior array panels contribute overwhelmingly to the overall SNR in the fetal brain. Acceleration
capability in the anterior-posterior direction was not substantially improved beyond what would be expected with just
the two views of the phantom provided by anterior and posterior panels. The side wings do, however, benefit acceleration
in the right-left direction.

Due to delays in IRB approval and subject recruitment, the array has thus far only been tested on the purpose built
pregnant abdomen phantom.  The phantom and array were designed based on the same segmented models of a pregnant
volunteer, and so by design the coil fits the phantom very well. Since this close fit is one of the noted advantages of
the 22 week fetal coil, further testing on a varied population of pregnant volunteers is needed to fully assess any
performance improvement. Furthermore, practical considerations like patient comfort have not yet been evaluated.
