\chapter{Array Design}
\section{Physical Design}
The 22 week fetal array is designed to provide good surface area coverage and close physical fit on a range of body
types at 22 weeks of pregnancy.  It consists of a rigid posterior panel that attaches directly to the patient table and
a group of anterior and lateral panels that can be freely positioned on the patient. The two lateral panels are attached
to the anterior panel with hinges to provide a degree of freedom. The patient facing surfaces of these panels are
modelled after segmented images of a 22 weeks pregnant volunteer. The precise geometry of the hinged panel assembly was
arrived at through iterative fit tests on pregnant volunteers.
  
The coil formers and housings were designed in Rhinoceros (Robert McNeel \& Associates, WA, USA) and printed in
polycarbonate on a (PRINTER MODEL NUMBER) (Stratasys, Ltd., MN, USA).

\section{Array Construction}
The array consists of four distinct groups of coil elements. The posterior panel contains 24 loops, each with a diameter
of approximately 9cm. The anterior panel contains 20 loops, with a median loop diameter of 8cm and with several
outliersin the non-hexagonally tiled area. The two anterior panels contain 10 7cm loops each. The loops in each panel
are for the most part arranged in a hexagonal tiling pattern that allows each loop to be critically overlapped with all
of its neighbors,  thus minimizing inductive coupling between neighboring loops [CITE ROEMER]. The loop layout is shown
in detail in FIGURE.

Individual loops were constructed from 16 gauge tin plated copper wire, with bridges bent into the wires to allow them
to cross each other without touching. A schematic of the loop circuitry is shown in FIGURE. The following chapter
contains a detailed explanaion of the function of the loop circuit The elements inside the dashed green region exist on
a small FR4 circuit board termed the “feed point board” 
