\chapter{Introduction}
MRI is increasingly used in addition to ultrasound to evaluate potential fetal disorders in routine clinical practice.
Indeed, MRI provides a higher soft tissue contrast than ultrasound in the fetus and reproductive organs. MRI also has
the potential to provide valuable physiological information through spectroscopic and diffusion weighted imaging. In
practice, the problem of unpredictable and nonrigid fetal motion limits fetal MRI to fast single shot T2 weighted
sequences which have poor tissue contrast, low SNR, and cannot provide detailed physiological information.

High density arrays are needed to minimize acquisition times and maximize SNR. A standard fetal MRI protocol employs a
spine array and a flexible body array to reach a total of approximately 34 channels. (HOW MANY WOULD BE OPTIMAL, CITE
PRUSSMAN) This limits the acceleration factors that can be achieved. Additionally, these general purpose coils are often
unable to completely conform to the varied anatomy of pregnant patients, leaving some receive elements distant to the
abdomen.In this work, we designed, built, and tested a semi-adjustable anatomically shaped 64 channel array coil for
fetal imaging at 22 weeks of pregnancy on a 3T MAGNETOM Skyra system (Siemens Healthcare GmbH, Erlangen, Germany) .
